% --------------------------------------------------------
%                         ERIK PILLON                     
%                    erik.pillon@gmail.com                
%                                                         
%                       PROJECT STARTED:                  
%                          20/11/2017                     
%                                                          
% --------------------------------------------------------
% 
% this is the main file
% will be filled up as soon as possible

% !TeX encoding = UTF-8
% !TeX spellcheck = en_US
\documentclass[10pt]{article}


\usepackage{geometry}
\geometry{margin=75pt,%
	top=40pt ,bottom=60pt}
\usepackage{microtype}
\usepackage{floatflt}

\usepackage[backend=bibtex]{biblatex} %I need to use biblatex for some special features
\addbibresource{bib.bib}

% THE FOLLOWING ARE OTHER BEAUTIFUL FONTS
%\usepackage{palatino}
%\usepackage{fontawesome}
%\usepackage{fourier}

\usepackage{hyperref}

\usepackage{newtxtext}
\usepackage{newtxmath}

\author{Erik Pillon}
\date{\today}
\title{What's needed for doing \emph{good} Theoretical Physics}
\begin{document}
\maketitle
\begin{abstract}
Brief explanation of what this stuff is... 
\end{abstract}
\section*{Introduction}
This is a fairly detailed collection of all the subjects, concepts, theorems and tools I'we studied during my educational trip. There is no request of completeness as well as this "few" concepts must be intended as a leading guide for the student that is approaching mathematics and/or physics from scratch.
\section{Elementary Math}
\begin{itemize}
	\item Sum and Difference.
	\item Multiplication and Division.
	\item Fraction.
	\item Powers.
	\item Logarithm and exponential.
\end{itemize}

\section{Logic}
\begin{itemize}
	\item Unary operators: \emph{not}: \textlnot  ; tautology: $ \top $	 
	\item Binary operators: and, or, nor, nand, ...
	\item Equivalence Class
	\item Axiom of Choice
\end{itemize}

\section{Linear Algebra}
\begin{itemize}
	\item Cayley–Hamilton theorem
\end{itemize}

\section{Single-valued Real Analysis}
\begin{itemize}
	\item Topology, open and closed sets.
\end{itemize}

\section{Mechanics}

\section{Thermodynamics}

\section{Algorithms and Computations}
\begin{itemize}
	\item Computational cost.
	\item List, Stack, Arrays, Queries.
	\item Bubble sort, Merge Sort, Quick Sort.
\end{itemize}
\section{Multi-valued Analysis}
\begin{itemize}
	\item Partial derivatives, derivatives vector $ \nabla $.
	\item Curl and Divergence. 
\end{itemize}
\section{Numerical Analysis}
\section{Algebra}
Even if this (sometimes \emph{very}) abstract subject is considered almost selfcontained and useful only for very narrow fields purposes, I wanted to be very precises in the subsections in order to underline the extreme importance that Algebra and (Lie) Groups in general has assumed in the last few decades. The ones listed belows are nowaday unavoidable topics for future theoreticians willing to completely understand all the Symmetry Properties in Nature, from Special and General Relativity to Classical and Quantum Mechanics as well as Particle Physics. 
\subsection{Group Theory}
\begin{itemize}
	\item Cayley Diagram
	\item Generators
	\item Klein Group
	\item Cyclic Group
	\item Abelian Group
	\item Dehydral Group
	\item Coset
	\item Normal Subgroup
	\item Quotient Group
	\item Semidirect Product
	\item Group Representation 
	\item Irreducible Group Class
\end{itemize}
\section{Geometry}
\begin{itemize}
	\item Topological space
	\item trivial topology
	\item discrete topology
	\item cofinite topology
	\item Neighbourhoods
	\item Closure
	\item Continuous applications
	\item Homeomorphisms
	\item Limit points and isolated points
	\item Dense set
	\item Topological subspace
	\item induced topology
	\item Product spaces
	\item Separation axioms
	\item Hausdorff spaces
	\item Normal spaces
	\item Regular spaces
	\item Countability axioms
	\item Quotient space
	\item Open and closed applications
	\item Relevant examples: sphere, projective space, Moebius strip
	\item Compactness
	\item Heine-Borel Theorem
	\item Tychonoff Theorem
	\item Bolzano-Weierstrass Theorem
	\item Connectivity, local connectivity
	\item Path connectivity
	\item Simply connected
	\item Homotopy and fundamental group
	\item Jordan curve Theorem
	\item Embedding and immersion. 
	\item Vector fields along a curve
	\item Tangent vector and line
	\item Length of an arc
	\item Parametrization by arc-length
	\item Inflection points
	\item Curvature and radius of curvature
	\item Center of curvature
	\item Frenet-Serret formula
	\item Tangent line
	\item Normal plane.
	\item Inflection points.
	\item Osculator plane.
	\item Curvatures.
	\item Principal frame.
	\item Frenet-Serret formula.
	\item Torsion.
	\item Fundamental Theorem.
	\item Differentiable atlas.
	\item Oriented atlas
	\item Tangent plane
	\item Normal versor.
	\item First fundamental quadratic form: metric and area.
	\item Tangential curvature and normal curvature of a curve on a surface.
	\item Curvatures
	\item normal sections
	\item Meusnier Theorem.
	\item Principal curvatures
	\item Gaussian curvature and mean curvature: Theorem Egregium.
	\item Geodetics. 
\end{itemize}

\section{Probability}

\section{Dynamical Systems}

\section{Electromagnetism}

\section{Fluid Dynamics}

\section{Wave Mechanics}
\begin{itemize}
	\item Wave Equation $\partial_{tt} u-c^2\Delta u=0$
	\item Planar wave
	\item Poynting Vector
\end{itemize}

\section{Complex Analysis}

\section{Numerical Analysis for (Partial) Differential Equations}

\section{Stochastic Processes}
\begin{itemize}
	\item Conditional expectations and conditional laws 
	\item Filtered probability space, filtrations 
	\item Adapted stochastic process (wrt a given filtration) 
	\item Martingale (Markov chains) 
	\item Kolmogorov characterization theorem 
	\item Stopping times 
	\item Definition of martingale process, resp. super, resp. lower, martingale 
	\item Stopping times for martingale processes 
	\item Convergence theorems for martingales 
	\item Markov chains (MC) 
	\item Transition matrix for a MC 
	\item Construction and existence for MC 
	\item Omogeneous MC (with respect to time and space)
	\item Canonical MC 
	\item Classification of states for a given MC (and associated classes) 
	\item Chapman-Kolmogorov equation 
	\item Recurrent, resp. transient, states ( classification criteria ) 
	\item Irriducible and recurrent chains 
	\item Invariant (stationary) measures, ergodic measures, limit measures ( Ergodic theorem ) 
	\item Birth and death processes (discrete time) 
	\item Continuous time MC 
	\item Absolute and stationary distributions 
	\item Probability and rates of transition 
	\item Kolmogorov differential equations 
	\item Stationary laws 
	\item Birth and death processes (first steps in continuous time) 
	\item Queuing theory (first steps in continuous time) 
	\item Point, Counting and Poisson Processes 
	\item Stochastic point processes (SPP) and Stochastic Counting Processes (SCP) 
	\item Stationarity, intensity and composition for SPP and SCP 
	\item Homogeneous Poisson Processes (HPP) 
	\item Non Homogeneous Poisson Processes (nHPP) 
	\item Mixed Poisson Processes (MPP) 
	\item Birth and Death processes (B\&D) 
	\item Time-dependent state probabilities 
	\item Stationary state probabilities 
	\item Inhomogeneous B\&D processes 
\end{itemize}
\begin{refsection}
	\nocite{baldi2007calcolo}
	\nocite{beichelt2006stochastic}
	\printbibliography[heading=subbibliography]
\end{refsection}
\section{Differential Geometry}
\begin{itemize}
	\item Differentiable Atlas
	\item Orientable Atlas
	\item Tangent plane
	\item Normal versor
	\item First Fundamental Form: lengths and area
	\item Geodesic curvature and normal curvature
	\item Normal sections and Meusnier Theorem
	\item Principal Curvatures, Gaussian curvature, Mean curvature: minimal surfaces
	\item Theorema Egregium
	\item Geodetics
	\item Free vector space
	\item Tensor product of two vector spaces
	\item Tensor product of n vector spaces
	\item Tensor Algebra
	\item Transformation of the componenents of a tensoriale
	\item Mixed tensors
	\item Symmetric tensors
	\item Antysimmetric (alternating) tensors
	\item Exterior Algebra
	\item Determinant
	\item Area and Volume
	\item Definition and examples
	\item Classification of 1-manifolds
	\item Classification of simply-connected 2-manifolds
	\item Product and quotient spaces
	\item Differentiable maps
	\item Tangent space and tangent bundle
	\item Vector field on a manifold
	\item Tensor field
	\item Exterior Algebra on manifolds
	\item Riemannian Manifolds
	\item Metric Tensor
	\item Orientations
	\item Volume
	\item Exterior derivative
	\item De Rham Cohomology
	\item Homotopy
	\item Affine connection
	\item Parallel transport
	\item Levi-Civita connection
	\item Geodetics
	\item Riemann curvature tensor
	\item Bianchi identities 
\end{itemize}
\begin{refsection}
	\nocite{do2017differential}
	\nocite{hou1997differential}
	\printbibliography[heading=subbibliography]
\end{refsection}

\section{Functional Analysis}
\begin{refsection}
\begin{itemize}
	\item $ L^p $ spaces
	\item Riesz Lemma
	\item Fredholm Alternative
\end{itemize}
\nocite{brezis2010functional}
\printbibliography[heading=subbibliography]
\end{refsection}

\section{Mathematical tools for Physics}
\begin{itemize}
	\item Eigenfunctions for the cube and for the cylinder
	\item Bessel Equation
	\item Bessel Functions of the first and second kind: $ J_{\alpha}(x) $ and $ Y_n(x) $
	\item Fourier-Bessel Series
	\item Eigenfunction for the sphere
	\item Laplacian in Spherical Coordinates
	\item Legendre Equation
	\item Legendre Polynomials
	\item Rodriguez Formula 
	\item Fourier-Legendre Series
	\item Recurrence Relations
	\item Associated Legendre Functions
	\item Spherical harmonics $ Y_{lm}(\theta,\varphi) $
	\item Perturbation Theory
	\item Fourier Transform in $\mathbb{R}^n$
	\item Green Function(s)
	\item Spectral Representation of Green (homogeneous) Functions
	\item $1^{st}$ and $2^{nd}$ Green Formulas
	\item Laguerre Polynomials
\end{itemize}

\section{Statistical Mechanics}
\begin{itemize}
	\item Fundamental Assumption of Equilibrium Statistical Mechanics
	\item Accessible Macrostate
	\item Liouville Equation $\frac{\partial \rho}{\partial t}=\{H,\rho\}$
\end{itemize}
\begin{refsection}
	\nocite{kittel1998thermal}
\printbibliography[heading=subbibliography,title={Recommended Books}]
\end{refsection}

\section{Solid State Physics}
\begin{itemize}
	\item Boltzman Model
	\item Einstein Model
	\item Bebye Model
	\item Drude Theory
\end{itemize}
\begin{refsection}
	\nocite{kittel2005introduction}
\printbibliography[heading=subbibliography,title={Recommended Books}]
\end{refsection}

\section{Nuclear Physics}

\section{Partial Differential Equations}
\begin{itemize}
	\item Characteristics Method
\end{itemize}
\begin{refsection}
	\nocite{salsa2016equazioni}
	\nocite{salsa2016partial}
\printbibliography[heading=subbibliography]
\end{refsection}

\section{Stochastic Differential Equations}
\begin{itemize}
	\item It\^o Integral
	\end{itemize}
\section{Advanced Numerical Analysis}
\section{Analytical Mechanics}
\section{Quantum Mechanics}
\begin{itemize}
	\item Schr\"odinger Equation
	\item Probability Density $\partial_t \psi^*\psi$
	\item Probability Current Density $\vec{\nabla}\cdot\vec{S}$
	\item Infinity conditions for the wave function
	\item Stationary States for a quantum mechanical system
	\item Klein-Gordon Equation
	\item Schr\"odinger Solution as a Markov process
	\item Simple Harmonic Oscillator
	\item Ladder Operators $a$ and $a^{\dag}$
	\item Hermite Differential Equation
	\item 1D Square Well Potential
	\item Forbidden Regions
	\item Square Potential Barriers
	\item Tunneling Effect
	\item Particle in the box
	\item Concept of classical limit $\hbar \to 0$
	\item Gauge Transformations and Landau Gauge
	\item Landau Levels
	\item Spherical Harmonics
	\item Pseudo-vectors/Axial Vectors
	\item Spin-Orbit Coupling
	\item Shell Model of the Nucleus
	\item Loosely bound states
	\item Isopartner fermions
	\item Generalized Pauli Principle
	\item Interacting Harmonic Oscillators
	\item Schr\"odinger Equation on a circle
	\item Dinamics of a particle in a box
	\item Schr\"odinger Picture
	\item Heisenberg Picture
\end{itemize}


\section{Nonequilibrium Statistical Mechanics}
Nonequilibrium statistical mechanics deals with the issue of microscopically modelling the speed of irreversible processes that are driven by imbalances. Examples of such processes include chemical reactions or flows of particles and heat. Unlike with equilibrium, there is no exact formalism that applies to non-equilibrium statistical mechanics in general, and so this branch of statistical mechanics remains an active area of theoretical research.
\begin{itemize}
	\item Einstein-Smoluchowski relation $D = \mu k_B T$
	\item Stokes-Einstein equation $D=\frac{k_{\mathrm{B}}T}{6\pi \,\eta \,r} $
	\item Ornstein-Uhlenbeck process
	\item Green–Kubo relations for transport coefficients $\gamma$
	\item Diffusion Tensor $D_{ij}$ 
\end{itemize}
\section{Advanced Quantum Theory}
\begin{itemize}
	\item Coupling Basis
	\item Clebsch-Gordan Coefficients
	\item Isospin
	\item Coherent State
	\item Displacement Operator
	\item Squeezing Operator
	\item Cross section amplitude coefficient $\sigma$
\end{itemize}

\section{Quantum Field Theory}

\section{Advanced Quantum Field Theory}

\end{document}


